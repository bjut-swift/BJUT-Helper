\documentclass[twocolumn,zihao=5,linespread=1,heading=false,autoindent=0pt]{ctexart}
\usepackage[T1]{fontenc}
\usepackage[a4paper,top=1cm,bottom=1cm,left=1cm,right=1cm,marginparwidth=1.75cm]{geometry}
\usepackage{mathtools}
\usepackage{tikz}
\usepackage{booktabs}
\usepackage{caption}
\usepackage{outlines}
\usepackage{graphicx}
\usepackage{amssymb}
\usepackage{float}
\usepackage{amsthm}
\usepackage{enumitem}
\usepackage{titlesec}
\usepackage{cancel}
\usepackage[colorlinks=false, allcolors=blue]{hyperref}
\renewcommand{\tableautorefname}{表}
\DeclarePairedDelimiter{\set}{\{}{\}}
\DeclarePairedDelimiter{\paren}{(}{)}
\graphicspath{ {./images/} }

\makeatletter
\DeclareRobustCommand{\em}{%
  \@nomath\em \if b\expandafter\@car\f@series\@nil
  \normalfont \else \heiti\bfseries \fi}
\makeatother

\usetikzlibrary{automata,positioning}
\newcommand{\hl}[1]{\colorbox{yellow}{#1}}
\newcommand{\tto}{\Rightarrow}
\pagestyle{empty}
\newenvironment{citemize}%
{\begin{itemize}[parsep=0pt,itemsep=0pt,topsep=0pt,partopsep=0pt]}
{\end{itemize}}
\newenvironment{cenumerate}%
{\begin{enumerate}[parsep=0pt,itemsep=0pt,topsep=0pt,partopsep=0pt]}
{\end{enumerate}}
\linespread{1}
\setlength{\parindent}{0pt}
\setlength{\parskip}{0pt}
\setlength{\baselineskip}{0pt}
\setlength{\abovedisplayskip}{0pt}
\setlength{\belowdisplayskip}{0pt}
\setlength{\abovedisplayshortskip}{0pt}
\setlength{\belowdisplayshortskip}{0pt}
\titlespacing*{\section}{0pt}{0pt}{0pt}
\titlespacing*{\subsection}{0pt}{0pt}{0pt}
\titlespacing*{\subsubsection}{0pt}{0pt}{0pt}

\setlength{\floatsep}{0pt plus 2pt minus 2pt}
\setlength{\textfloatsep}{0pt plus 2pt minus 2pt}
\setlength{\intextsep}{0pt plus 2pt minus 2pt}
\newcommand{\includedrawio}[2][]{
    \immediate\write18{/Applications/draw.io.app/Contents/MacOS/draw.io #2 --crop -x -o #2.pdf}
    \includegraphics[#1]{#2.pdf}
}

\ctexset{
    section = {
        titleformat = \raggedright,
        format += \small,
        name = {,},
    },
    subsection/format += \small,
    subsubsection/format += \small,
    paragraph = {
        runin = false
    },
    today = small,
    figurename = 图,
    contentsname = 目录,
    tablename = 表,
}

\newtheoremstyle{exampstyle}
  {0pt} % Space above
  {0pt} % Space below
  {} % Body font
  {} % Indent amount
  {\bfseries} % Theorem head font
  {} % Punctuation after theorem head
  {2ex} % Space after theorem head
  {} % Theorem head spec (can be left empty, meaning `normal')

\theoremstyle{exampstyle} \newtheorem{definition}{定义}[section]
\theoremstyle{exampstyle} \newtheorem{example}{例}[section]
\theoremstyle{exampstyle} \newtheorem{theorem}{定理}[section]
\theoremstyle{exampstyle} \newtheorem{lemma}{引理}[section]
\theoremstyle{exampstyle} \newtheorem{myproof}{证明}[section]

\begin{document}
\section{语言与文法}
\footnotesize

句子的逆 $x^R$ or $x^T$.

\begin{definition}
    文法(Grammar)G是一个四元组
    $
        G = (V, T, P, S)
    $
    其中,

    V--- \emph{变量(Variable)}的非空有穷集。$\forall A \in V$,
    $A$叫做语法变量(syntactic variable),也叫非终极符号(nonterminal)。
    
    T--- \emph{终极符(Terminal)}的非空有穷集。$\forall a \in T$,
    $a$叫做终极符。 $V \cup T = \emptyset$。

    P--- \emph{产生式(Production)}的非空有穷集。对于$a \to b$,
    $a$是\emph{左部},$b$是\emph{右部}。

    S--- $S \in V$,文法G的\emph{开始符号(Start symbol)}。
\end{definition}

\begin{outline}[citemize]
    \1 只写产生式,第一个产生式的左部为开始符号
    \1 对一组有相同左部的产生式 \\
        $\alpha \to \beta_1, \alpha \to \beta_2, \alpha \to \beta_3, \dots$
        可以记为 $\alpha \to \beta_1 | \beta_2 | \beta_3 \dots$
        $\beta_1 , \beta_2 , \beta_3$称为\emph{候选式}(Candidate)
    \1 形如$\alpha \to \epsilon$的产生式叫做空产生式,也叫做$\epsilon$产生式
        \2 英文大写字母为\emph{语法变量}
        \2 英文小写字母为\emph{终结符号}
        \2 英文较后的大写字母为\emph{语法变量或者终极符号}
        \2 英文较后的大写字母为\emph{终极符号行}
        \2 希腊字母表示\emph{语法变量和终极符号组成的行}
\end{outline}

\begin{definition}
    设$G = (V,T,P,S)$是一个文法,如果
    $\alpha \to \beta\in P, \gamma,\delta  \in (V \cup T)$,则称
    $\gamma\alpha\delta$在$G$中\emph{直接推导}(Derivation)出$\gamma\beta\delta$,记作
    $\gamma\alpha\delta \underset{G} \Rightarrow \gamma\beta\delta$。

    于此相对应,$\gamma\beta\delta$归约到$\gamma\alpha\delta$,简称$\beta$归约为$\alpha$。

    $\underset{G} \Rightarrow$ 是$\left( V \cup T \right)^*$上的二元关系。
\end{definition}

\begin{definition}对于文法$G$:
    $\underset{G}{\overset{n}\Longrightarrow } = \left(\underset{G} \Rightarrow \right)^n $ 
    $\underset{G}{\overset{*}\Longrightarrow } = \left(\underset{G} \Rightarrow \right)^* $
    $\underset{G}{\overset{=}\Longrightarrow } = \left(\underset{G} \Rightarrow \right)^+ $
\end{definition}

\begin{definition}
    对于语言$G = (V, T, P, S)$:

    语法范畴A $L(A) = \set*{w | w \in T^* \text{且} A \overset{*}\Rightarrow w}$

    语言(Language) $L(G) = \set*{w | w \in T^* \text{且} S \overset{*}\Rightarrow w}$

    句子(Sentence) $\forall w \in L(G)$

    句型(Sentential Form) $\forall \alpha \in (V \cup T)^*$, 
    如果$S \overset{*}\Rightarrow \alpha$,则称$\alpha$是$G$产生的一个句型。
    
\end{definition}

\begin{definition}
    对于文法 $G_1, G_2$,如果$L(G_1) = L(G_2)$,则称$G_1$与$G_2$等价。
\end{definition}

\subsection{文法的乔姆斯体系}
\begin{definition}
    对于文法$G = (V,T,P,S)$: \\
    G叫做0型文法,也叫短语结构文法(PSG,Phrase Structure Grammar) \\
    $L(G)$是0型语言,也叫短结构语言,可递归枚举集。
\end{definition}
\begin{definition}
    对于0型文法文法$G = (V,T,P,S)$: \\
    $\forall \alpha \to \beta \in P$,$|\beta| \geq |\alpha|$,
    则G是1型文法,或上下文有关文法。
\end{definition}
\begin{definition}
    对于1型文法文法$G = (V,T,P,S)$: \\
    $\forall \alpha \to \beta \in P$,$|\beta| \geq |\alpha|$,
    $\alpha \in V$
    则G是2型上下文无关文法。
\end{definition}
\begin{definition}
    对于2型文法文法$G = (V,T,P,S)$: 
    $\forall \alpha \to \beta \in P$: 
    $A \to wB$和$A \to w$,其中$A,B \in V, w \in T^+$:G是右线性文法。 
    $A \to Bw$和$A \to w$,其中$A,B \in V, w \in T^+$:G是左线性文法。 
    则G是3型文法,或正则文法。
\end{definition}

\subsection{空产生式}
允许在CSG,CFG,RG文法中存在空产生式。 

允许在CSL,CFL,RL语言中存在空语句。

左右线性文法等价,其中左线性的表述好。\emph{左右都有的线性文法不是正则文法}。

$\forall G, \exists G', L(G') = L(G)$,但是$G'$中的开始符号不出现在任何
    产生式的右部,且在$\epsilon \in L(G')$时,$G'$中只有$S' \to \epsilon$这样一个
    $\epsilon$产生式。\emph{可以去S产生式和$\epsilon$产生式}

\subsubsection{语言运算}
给定上下文无关文法$G_1,G_2$,构造$G$使得:
\begin{outline}[cenumerate]
    \1 $L(G) = L(G_1)L(G_2)$ \\
        其中$V_1 \cup V_2 = \emptyset$, $S \notin V_1 \cup V_2$

        $G = (V_1 \cup V_2 \cup {S}, T, P_1 \cup P_2 \cup {S \to S_1S_2}, S)$

    \1 $L(G) = L(G_1) \cup L(G_2)$

        $G_\cup = (V_1 \cup V_2 \cup {S}, T, P_1 \cup P_2 \cup P_3, S)$ \\
        $P_3 = \set*{S \to S_1 | S_2}$

    \1 $L(G) = L(G_1)^*$

        $G_* = \paren*{V_1 \cup \set*{S}, T, P_1 \cup P_2, S}$ \\
        $P_2 = \set*{S \to \epsilon | SS_1}$

\end{outline}

给定RG $G_1,G_2$,构造RG $G$使得:
\begin{outline}[cenumerate]
    \1 $L(G) = L(G_1)L(G_2)$ \\
        其中$V_1 \cup V_2 = \emptyset$, $S \notin V_1 \cup V_2$

        $G = (V_1 \cup V_2 \cup {S}, T, P_1' \cup P_2 \cup P_3, S_1)$

        $P_1' = \set*{A \to a B | A \to a B \in P_1}$
        $P_3 = \set*{A \to a S_2 | A \to a \in P_1}$

    \1 $L(G) = L(G_1) \cup L(G_2)$

        $G_\cup = (V_1 \cup V_2 \cup {S}, T, P_1 \cup P_2 \cup P, S)$

        $P = \set*{S \to \alpha | S_1 \to \alpha \in P_1} \cup \set*{S \to \alpha | S_2 \to \alpha \in P_2}$

    \1 $L(G) = L(G_1)^*$

    $G_\cup = (V_1 \cup {S}, T, P_1 \cup P, S)$
    $\begin{aligned}
        P & = \set*{S \to \epsilon | S_1S}
    \end{aligned}$
\end{outline}
\begin{definition}
    $M = (Q, \Sigma, \delta, q_0, F)$,其中:

    \begin{description}[parsep=0pt,itemsep=0pt,topsep=0pt,partopsep=0pt]
        \item[$Q$]:状态的有穷集合
        \item[$\Sigma$]: 输入字母表
        \item[$\delta$]: 状态转义函数
            $Q\times\Sigma \to Q$。 $\forall (q,a) \in Q\times\Sigma$,
            $\delta(q,a) = p$
        \item[$q_0$]:开始状态
        \item[$F$]: 终止状态 
    \end{description}
\end{definition}
\begin{definition}
    扩展$\delta$为$\hat{\delta}: Q \times \Sigma^* \to Q$
    \begin{cenumerate}
        \item $\hat{\delta}(q,\epsilon) = q$
        \item $\hat{\delta}(q, wa) = \delta(\hat{\delta}(q,w), a)$
    \end{cenumerate}
    注意到 $Q \time \Sigma \subset Q \times \Sigma^*$,
    且对$\forall (q, a) \in Q \times \Sigma$,$\hat{\delta}(q, a) = \delta(q, a)$
    所以,不用区分$\delta$与$\hat{\delta}$。 \\
    $\delta$是$Q \times \Sigma^* \to Q$上的映射。
\end{definition}

\begin{definition}
    有穷状态自动机$M = (Q, \Sigma, \delta, q_0, F)$识别的语言:
    $L(M) = \set*{x | \delta(q_0, x) \in F}$
\end{definition}

\begin{definition}
    有穷状态自动机$M_1, M_2$:若有$L(M_1) = L(M_2)$,则称二者等价。
\end{definition}

\begin{definition}
    对于有穷自动状态机$M = (Q, \Sigma, \delta, q_0, F)$
    
    $set(q) = \set*{x | \delta(q_0, x) = q}$

    所以有:$\Sigma^* = \bigcup_{q \in Q}set(q)$

    $\forall q,p \in Q, q \ne p, set(q) \cap set(p) = \emptyset $

    同时,如果$Q$中不存在不可达状态,则$set(q_0), set(q_1) \dots$
    是$\Sigma^*$的一个划分。
\end{definition}

\begin{definition}
    $
    x R_M y \iff \delta(q_0, x) = \delta(q_0, y)
            \iff \exists q \in Q, x, y \in set(q)
    $
\end{definition}


\begin{example}
    对于FSA $M = (Q, \Sigma, \delta, q_0, F)$,构造FA 
    $M'$使得$L(M') = \Sigma^* - L(M)$ \quad$M' = (Q, \Sigma, \delta, q_0, \complement_QF )$
\end{example}

\begin{example}
    给定 $
        M_1 = (Q_1, \Sigma, \delta_1, q_{01}, F_1)\quad
        M_2 = (Q_2, \Sigma, \delta_2, q_{02}, F_2)
    $
    构造:
    \begin{outline}[citemize]
        \1 $M_3$使得$L(M_3) = L(M_1) \cap L(M_2)$


        $M = (Q_1 \times Q_2, \Sigma, \delta, [q_{01}, q_{02}], F_1 \times F_2 )$


        \1 $M_3$使得$L(M_3) = L(M_1) \cup L(M_2)$

        $M = (Q_1 \times Q_2, \Sigma, \delta, [q_{01}, q_{02}], Q_1 \times F_2 \cup F_1 \times Q_2)$

        $\delta([q, p], a) = [\delta_1(q, a), \delta_2(p, a)]$
    \end{outline}
\end{example}

\subsection{NFA 不确定的有穷状态自动机}
\begin{definition}
    NFA $M = (Q, \Sigma, \delta, q_0, F)$,其中:

    $Q, \Sigma, q_0, F$ 同 FA。

    $\delta$: $Q \times \Sigma \to 2^Q$ (幂集),
    $\delta(q, a) = \set*{p_1, p_2, \dots, p_n}$ 可选一个进入

    扩展$\delta$为$\hat{\delta}: Q \times \Sigma^* \to 2^Q$ ~~
    $
            \hat{\delta}(q,\epsilon)  = \set*{q} 
    $ ~~
    $
            \hat{\delta}(q, wa) = \bigcup_{q_0 \in \hat{\delta}(q,w)}\delta(q_0, a)
    $

    进一步拓展: $\hat{\delta}: 2^Q \times \Sigma^* \to 2^Q$
    $
        \hat{\delta}(P, x) = \bigcup_{p \in P}\hat{\delta}(p, x)
    $

    对$\forall (q, a) \in Q \times \Sigma^*$,$\hat{\delta}(q, a) = \delta(q, a)$
    所以不用区分$\delta$与$\hat{\delta}$。 

    $\delta$是$Q \times \Sigma^* \to 2^Q$上的映射。
\end{definition}


\begin{theorem}
    NFA与DFA等价。

    证:对于NFA $ M= (Q, \Sigma, \delta, q_0, F)$,构造
    DFA $M' = (2^Q, \Sigma, \delta', [q_0], F')$

    $F'= \set*{[q_1, q_2, \dots, q_n] |
     \set*{q_1, q_2, \dots, q_n} \cap F \ne \emptyset}$

    $\delta'([q_1, q_2, \dots, q_n], a) = 
    [\beta_1, \beta_2, \dots, \beta_n] \iff 
    \delta(\set*{q_1, q_2, \dots, q_n}, a) = \set*{\beta_1, \beta_2, \dots, \beta_n}
    $

    所以,对于$\forall M$ 是NFA,可以构造DFA $M'$
\end{theorem}

\subsection{带空移动的有穷状态自动机}

\begin{definition}
    $\epsilon-\mathrm{NFA}$ $M=(Q,\Sigma,\delta,q_0,F)$:

    $\delta$:状态转义函数。
    $\delta: Q \times (\Sigma \cup \set*{\epsilon}) \to 2^Q$

    其中,对于$\forall q \in Q, \delta(q, \epsilon)=\set*{p_1, p_2, \dots, p_n}$:
    表示在$q$状态不读入任何字符,可以将状态变为$p_1, p_2, \dots, p_n$,称为
    $M$在$q$状态做了一次空移动


    \begin{outline}[cenumerate]
        \1 $\epsilon-CLOSURE(q) = \set{p | \text{从q到p有一条标记为}
        \epsilon\text{的路}}$
        \1 $\epsilon-CLOSURE(P) = \bigcup_{p \in P}\epsilon-CLOSURE(p)$
        \1 $\hat{\delta}(q, \epsilon) = \epsilon-CLOSURE(q)$
        \1 $\hat{\delta}(q, wa) = \epsilon-CLOSURE(P)$ \quad $P = \bigcup_{r \in \hat{\delta}(q,w)}\delta(r, a)$

            
        \1 进一步拓展$\delta: 2^Q \times \Sigma \to w^Q$: \quad$\delta(P, a) = \bigcup_{q \in P}\delta(q, a)$

        \1 进一步拓展$\hat{\delta}: 2^Q \times \Sigma^* \to 2^Q$: \quad$\delta(P, w) = \bigcup_{q \in P}\hat{\delta}(q, w)$

            \emph{注意:$\hat{\delta} \ne \delta$, $\delta(q, a)$只要a不是$\epsilon$,就不能走空移动。 反过来,$\hat{\delta}(q,w)$在w字符串的中间可以走任意个空移动}
    \end{outline}
\end{definition}

\begin{definition}
    $\epsilon-NFA$ $M$识别的语言:
    $L(M) = \set{x|\hat{\delta}(q_0, x) \cap F \ne \emptyset}$
\end{definition}

\begin{theorem}
    $\epsilon-NFA$ 等价与 $NFA$。

    证明:设$\epsilon-NFA$ $M = (Q, \Sigma, \delta, q_0, F)$:

    取$M' = \left(Q, \Sigma, \hat{\delta}, q_0, \begin{cases}
        F \cup \set*{q_0} & F \cap \epsilon-CLOSURE(q_0) \ne \emptyset \\
        F & F \cap \epsilon-CLOSURE(q_0) = \emptyset
    \end{cases}\right)$
\end{theorem}

由此可见,$DFA$,$NFA$,$\epsilon-NFA$等价。以后统称为FA。

\subsection{FA与RG等价}

\begin{theorem}
    对$\forall\;DFA\;M,\;\exists\;RG\;G$使得$L(G) = L(M)$。

    构造:对于$DFA\; M=(Q,\Sigma,\delta,q_0, F)$,取RG
    $G=(Q, \Sigma, P, q_0)$

    $\begin{aligned}
        P &= \set*{q \to ap | \delta(q, a) = p} \cup 
    \set*{q \to a | \forall \delta(q, a) \in F} \cup
    \set*{q_0 \to \epsilon | q_0 \in F} \\
        & = \set*{q \to ap | \delta(q, a) = p} \cup 
        \set*{q \to \epsilon | q \in F} \\
    \end{aligned}$
\end{theorem}

\begin{theorem}
    对$\forall\;\text{RG}\;G,\;\exists\; \text{FA} \;M$使得$L(G) = L(M)$。

    构造:对于RG $G=\paren*{V, T, P, S}$,RG为右线性文法。

    取FA $M=(V \cup \set*{f},T,\delta,S, \set*{f})$,其中

    $\delta(A, a) = \set*{B | \forall A \to aB \in P} \cup 
    \set*{f | \forall A \to a \in P} 
    $
\end{theorem}

\section{正则表达式}

\begin{definition}
    $\sigma$上的RE是满足如下条件的式子:
    \begin{outline}[citemize]
        \1 $\epsilon$ 是RE,表示的语言$L(\epsilon) = \set*{\epsilon}$
        \1 $\emptyset$ 是RE,表示的语言$L(\emptyset) = \emptyset$
        \1 $\forall a \in \sigma,\; a$是RE,表示的语言$L(a) = \set{a}$
        \1 $\forall r, s$是RE且分别表示$R,S$,则有:
            \2 $(r+s)$是RE,表达的语言$L(r+s) = R \cup S$
            \2 $(rs)$是RE,表达的语言$L(rs) = RS$
            \2 $(r^*)$是RE,表达的语言$L(r^*) = R^*$
    \end{outline}
    约定1:运算符优先级$* > \times > +$ \\
    约定2:引入正闭包$r^+ = r^*r$
\end{definition}

\subsection{RE \textrightarrow FA}

\begin{figure}[H]
    \centering
    \includedrawio[width=0.6\columnwidth]{re-to-fa.drawio}
\end{figure}
\begin{theorem}
    对于$\forall$ RE $r, \exists$ FA $M$,使得$L(M) = L(r)$
\end{theorem}

\section{正则语言的性质}
\subsection{正则语言的泵引理}
\begin{lemma}
    正则语言的的泵引理。对于$\forall\; RL\; L$,存在一个仅依赖与$L$的正整数$N$,对于$\forall z
    \in L, |z| \ge N$,则存在$u, v, w$满足以下条件:

    \centerline{
        $uvw = z$ \quad
        $|uv| \le N$ \quad
        $|v| \ge 1$ \quad
        对于$\forall k \ge 0$,$uv^kw \in L$
    }
\end{lemma}

\begin{lemma}
    拓展的泵引理。对于$\forall\; RL\; L$,存在一个仅依赖与$L$的正整数$N$,对于$\forall z = z_1z_2z_3
    \in L, |z_2| \ge N$,则存在$u, v, w$满足以下条件:

    \centerline{$uvw = z_2 \quad |uv| \le N \quad |v| \ge 1 \quad \text{对于}\forall k \ge 0, uv^kw \in L$}
\end{lemma}

\begin{example}
    证明$L=\set{0^p|p\text{是质数}}$不是RL。

    设$L$是RL。$N$是泵引理所说正整数。设$x$为大于$N$的最小质数。

    取$z = 0^x$。取$v = 0^l,\; l \ge 1$。$uv^kw = 0^{x+(k-1)l}$。

    当$k = x+1$时,有$x+(x+1-1)l = (l+1)x$是合数,所以$uv^kw \notin L$,与泵引理矛盾。所以L不是RL。
\end{example}

\begin{theorem}
    RL对于并、乘、闭包、交、补封闭。
\end{theorem}

\subsection{Myhill定理}
\begin{definition}
    $\begin{aligned}[t]
        x R_M y & \iff \delta(q_0, x) = \delta(q_0, y) \\
            & \iff \exists q \in Q, x, y \in set(q)
    \end{aligned}$
\end{definition}
\begin{definition}
    $\begin{aligned}[t]
        x R_L y & \iff \forall z \in \Sigma^*, xz \in L \iff yz \in L
    \end{aligned}$
\end{definition}
\begin{definition}
    R是右不变的是指如果$xRy$则有$\forall z \in \Sigma^*, xzRyz$
\end{definition}
\begin{theorem}
    对于DFA M,如果$x R_M y$,则$x R_{L(M)} y$。
    
    证明:由$x R_M y$知$\delta(q_0, x) = \delta(q_0, y)$。不妨设其为
    $q_1$。

    对于$\forall z \in \Sigma^*$,$\delta(q_0, xz) =  \delta(\delta(q_0, x), z) = \delta(q_1, z) 
    = \delta(q_0, yz)$。不妨设其为$q_2$。

    如果$xz \in L$,则有$q_2 \in F$,所以$yz \in L$。反之亦然。

    所以 $xz \in L \iff yz \in L$。
    所以 $x R_M y \iff x R_{L(M)} y$
    \qed
\end{theorem}

\begin{theorem}
    Myhill定理。以下3个命题等价:
    \begin{outline}[cenumerate]
        \1 L 是 RL
        \1 L 是 $\Sigma^*$上的某个具有有穷指数的右不变等价关系的某些等价类的并。
        \1 $R_L$ 具有有穷指数。
    \end{outline}

    \begin{myproof}
        $1 \Rightarrow 2$。 设L是RL,DFA M使得$L(M) = L$。

        $R_M$是$\Sigma^*$上的右不变等价关系,且$|\Sigma^* / R_M| \le |Q|$。

        $ L = \bigcup_{q \in F}set(q)$
    \end{myproof}

    \begin{myproof}
        $2 \Rightarrow 3$。 设$L$是$\Sigma^*$上的具有有穷指数的右不变等价关系$R$的某些等价类的并。

        下面证明如果$x R y$,那么$x R_L y$。

        设$x, y \in \Sigma^*, xRy$。由$R$的右不变性,对于$\forall z \in \Sigma^*$,$xzRyz$。

        再注意到$L$是$R$的某些等价类的并,且$xz,yz$在同一个等价类中,所以$xz \in L \iff yz \in L$。
        
        所以$x R_L y$。
    \end{myproof}

    \begin{myproof}
        $3 \Rightarrow 1$。 设$R_L$具有有穷指数。取$M' = (\Sigma^*/R_L, \Sigma, \delta', [\epsilon], \set{[x] | x \in L})$

        $\delta'([a], x) = [ax]$。显然,$L(M') = L$。
    \end{myproof}
\end{theorem}

\hl{意义:L是RL的充要条件}

\begin{theorem}
    在同构意义下,$M'$是状态最少的唯一识别L的DFA。
\end{theorem}

\subsubsection{证明L 是/不是 RL}
\begin{outline}[cenumerate]
    \1 证明L是RL。 证明$R_L$的指数有穷。 \emph{列举所有等价类即可}
    \1 证明L不是RL。 证明$R_L$的指数无穷。
\end{outline}
\begin{example}
    证明 $\set{0^n1^n | n \ge 0}$ 不是RL。

    易得,对于$\forall i \ne j \in N$,$0^i \cancel{R_L} 0^j$
    因此$R_L$的指数无穷。
\end{example}

\subsection{DFA的最小化}
标记终止状态和非终止状态不能合并 \\
逐个判断其他的能否合并

\section{CFL 上下文无关语言}
\subsection{CFG}
\begin{definition}
    CFG $G=(V, T, P, S)$的语法树为满足如下条件的树:

    \begin{outline}[cenumerate]
        \1 每个节点的标记$x \in V \cup T \cup \set{\epsilon}$
        \1 如果节点V的标记为$A$,V从左到右的子节点$V_1,V_2,\dots,V_k$
        的标记依次为$y_1,y_2,\dots,y_k$,则$A \to y_1y_2\dots y_k \in P$
        \1 根节点标记为S
        \1 中间节点的标记为变量$x \in V$
        \1 从左到右的叶子节点$v_1,\dots,v_n$的标记$x_1,\dots,x_n$
        组成的串$x_1x_2\dots x_n$为该树的结果。
        \1 如果v的标记为$\epsilon$,则它没有兄弟。
    \end{outline}
\end{definition}
\begin{definition}
    满足语法树定义中除第三条外条件的树,称作A-子树。
\end{definition}
\begin{theorem}
    有一颗结果为$\alpha$的语法树$\iff S \overset{*} \Rightarrow \alpha $
\end{theorem}
\begin{definition}
    每一步派生均实施在当前句型最右变量上的派生叫最右派生。

    每一步派生均实施在当前句型最左变量上的派生叫最左派生。
\end{definition}
\begin{theorem}
    最左派生与最右派生的语法树是一一对应的。
\end{theorem}
\begin{definition}
    如果CFG G有句子有棵颗不同的语法书,则G是二义性的。
\end{definition}
\begin{definition}
    如果CFL L没有非二义性文法,则称之为固有二义性的。
\end{definition}
\subsection{去无用符号}

\subsubsection{去除无用符号}
\begin{definition}
    X是有用符号,即$\exists X \in L(G)$,$S \overset{*} \Rightarrow
    \alpha X \beta \overset{*} \Rightarrow x$。

    X是有用的,必须同时满足如下两条:
    \begin{outline}[cenumerate]
        \1 $S \overset{*} \Rightarrow \alpha X \beta$
            
            如何判断?

            \2 $V' = \set{S} \cup \set{A | S \to \alpha A \beta \in P}$

                $T' = \set{a | S \to \alpha a \beta \in P}$
            \2 重复 \\
            $V' = V' \cup \set{B | A \to \alpha B \beta \in P \& A in V'}$ \\
            $T' = T' \cup \set{a | A \to \alpha a \beta \in P \& A in V'}$
        \1 $X \overset{*} \Rightarrow w, w \in T^*$
        
            $G = (V, T, P, S)$,对于:

            $\forall a \in T$,$a$满足2。
        
            $\forall A \in V$,如何判断$A$是否满足2?
            % \begin{outline}[cenumerate]
                \2 $V' = \set{A | A \to w \in P}$
                \2 重复 $V' = \set{A | A \to \alpha, \alpha \in (V' \cup T)^*} \cup V'$
        % \end{outline}
    \end{outline}
\end{definition}
\begin{theorem}
    对于$\forall$ CFG G,$\exists$ CFG $G'$,
    \begin{outline}[cenumerate]
        \1 $L(G') = L(G)$
        \1 $G'$中无无用符号。
    \end{outline}
\end{theorem}


\subsubsection{去除$\epsilon-$产生式}
\begin{definition}
    如果$A \overset{*} \Rightarrow \epsilon$,则称A为可空变量。
\end{definition}
\begin{theorem}
    对于$\forall$ CFG $G, G'$:
    \begin{outline}[cenumerate]
        \1 $G'$中无空产生式
        \1 $L(G') = L(G) - \set{\epsilon}$
    \end{outline}
    对于$A \to x_1x_2\dots x_n$,替换为$A \to y_1y_2\dots y_n$,其中
    当$x_i$不是可空变量时,$y_i = x_i$,否则$y_i = x_i$或$\epsilon$。

    注意$y_1y_2\dots y_n$不能都为$\epsilon$。
\end{theorem}

\subsubsection{去单一产生式}
\begin{definition}
    形如$A \to B$的产生式是单一产生式。
\end{definition}
\begin{definition}
    对$\forall$ CFG G,$\exists$ CFG $G', L(G') = L(G)$,$G'$中无单一产生式。

    推论:对于$\forall$ CFG $G$, 存在$CFG$ G' 使得$L(G') = L(G)$,且$G'$没有无用符,$\epsilon -$产生式和单一产生式。
\end{definition}

\subsection{CNF 乔姆斯基范式}
\begin{definition}
    如果G的产生式均具有以下形式:
    
    $
    \begin{cases}
        A \to BC \\
        A \to a
    \end{cases}
    $

    则称之为CNF。
\end{definition}
\subsection{GNF 格雷巴赫范式}
\begin{definition}
    如果G的产生式均具有以下形式:
    
    $
        A \to a\alpha, \alpha \in V^*
    $

    则称之为GNF。
\end{definition}

\begin{example}
    $A \to Aa | Ab | c | d = \begin{cases}
        A \to c | d | cB | dB \\
        B \to a | b | aB | bB
    \end{cases}$
\end{example}
\begin{example}
    $
        A \to  A\alpha_1 | A\alpha_2 | A\alpha_3 | \beta_1 | \beta_2 
        = \begin{cases}
            A \to \beta_1 | \beta_2 | \beta_1B | \beta_2B \\
            B \to \alpha_1 | \alpha_2  | \alpha_3 | \alpha_1B | \alpha_2B | \alpha_3B
        \end{cases}
    $
\end{example}

\begin{example}
    $
        \begin{cases}
            S \to ABS | BAA \\
            A \to BB | BAA \\
            B \to aB | b
        \end{cases} = \begin{cases}
            S \to aBBBS | bBBS | aBAABS | bAABS 
                    aBAA | bAA\\
            A \to aBB | bB | aBAA | bAA \\
            B \to aB | b
        \end{cases}
    $
\end{example}

步骤:
\begin{outline}[cenumerate]
    \1 给变量排序
    \1 从$A_1$到$A_n$逐一使产生式满足如下要求:

    $A_i \to A_h \alpha, j \ge i$
    \1 从$A_{n-1}$开始通过回代,逐一使$A_{n-1},A_{n-2}\dots$的产生式满足要求
    \1 通过代入,使第二步中引入的新变量的产生式满足要求。

    关键: \emph{去左递归}

    如:

    $
    \begin{cases}
        A \to A\alpha_1 | A\alpha_2 | \dots | A\alpha_n \\
        A \to \beta_1 | \beta_2 | \dots | \beta_m
    \end{cases}
    $
    为所有A产生式,且$\beta_1, \beta_2, \dots, \beta_m$的首字母不是A,可以用如下的产生式组替代:
    
    $
    \begin{cases}
        A \to \beta_1 | \beta_2 | \dots | \beta_m | \beta_1 A' | \beta_2 A' | \dots | \beta_, A' \\
        A' \to \alpha_1 A' | \alpha_2 A' | \dots | \alpha_n A' | \alpha_1 | \alpha_2 | \dots | \alpha_n
    \end{cases}
    $
\end{outline}
\begin{theorem}
    对于$\forall$化简了的CFG,$\exists$ GNF 与之等价。
\end{theorem}

\section{PDA 下推自动机}
\subsection{PDA的定义}
\begin{definition}
    PDA $M = (Q, \Sigma, \Gamma, \delta, q_0, z_0, F)$
    其中:
    \begin{description}
        \item[$Q$]:状态的有穷集合
        \item[$\Sigma$]: 输入字母表
        \item[$\Gamma$]: 栈符号的非空有穷集
        \item[$\delta$]: 状态转义函数\\
            $Q\times (\Sigma \cup \set{\epsilon}) \times \Gamma \to 2^{Q \times \Gamma^*} $。 
            
            $\forall (q,a,A) \in Q\times (\Sigma \cup \set{\epsilon}) \times \Gamma$,
            $\delta(q,a,A) = \set{(p_1,\gamma_1),\dots,(p_k,r_k)}$表示$M$在状态$q$,栈顶为$A
            $时读到$a$,将栈顶符号$A$弹出,将$\gamma_i$依次压入栈,并将状态改为$p_i$。

            $\forall (q,A) \in Q\times \Gamma$,
            $\delta(q,\epsilon,A) = \set{(p_1,\gamma_1),\dots,(p_k,r_k)}$表示$M$在状态$q$,栈顶为$A
            $时做空移动,将栈顶符号$A$弹出,将$\gamma_i$依次压入栈,并将状态改为$p_i$。
        \item[$q_0$]:开始状态
        \item[$z_0 \in \Gamma$]:栈底符号
        \item[$F$]: 终止状态 
    \end{description}
\end{definition}

\begin{definition}
    PDA M的ID $(q, x, \alpha) = Q \times \Sigma^* \times \Gamma^*$

    其中 $q$ 是M的当前状态, $x$是M的输入带上剩余串,$\alpha$是M的栈中当前的内容。

    设M当前的ID是$(q, ax, A\alpha)$,如果$(p, \gamma) \in \delta(q, a, A)$,则M的ID变为$(p, x, \gamma \alpha)$,记作$(q, ax, A\alpha) \vdash_M (p, x, \gamma\alpha)$。

    如果$(p, \gamma) \in \delta(q, \epsilon, A)$,则M的ID变为$(p, ax, \gamma \alpha)$,记作$(q, ax, A\alpha) \vdash_M (p, ax, \gamma\alpha)$。
    
    $\vdash_M$ 是 $Q \times \Sigma^* \times \Gamma^*$上的二元关系。
\end{definition}

\begin{definition}
    $M = (Q, \Sigma, \Gamma, \delta, q_0, z_0, F)$

    用终态识别的语言$L(M) = {x | (q_0, x, z_0) \vdash^* (q, \epsilon, \alpha) \text{ and } q \in F}$。

    用空栈识别的语言$N(M) = {x | (q_0, x, z_0) \vdash^* (q, \epsilon, \epsilon) \text{ and } q \in F}$。
\end{definition}

\end{document}
